\subsection{Tailorable software}

In the context of End-User development an End-User is referred to as a person who is not an expert in Computer Science but uses a computer as a tool for performing activities relating to a given domain of expertise~\cite{lieberman2006}. With the development of technologies end-users have become more willing and demanding in terms of capability to tailor a software system to their own specific needs.\cite{Nardi1993}. The increasing desire of end users to become information producers, to be able to shape their software tools and to participate in the design of their own product moves the design paradigm of a software systems toward an evolutionary and never-ending process where end-users become co-designers.\cite{Carmelo2011}

\epigraph{End-User Development can be defined as a set of methods, techniques, and tools that allow users of software systems, who are acting as non-professional software developers, at some point to create, modify or extend a software artifact.
}{\textit{Lieberman et al.\cite{lieberman2006}}}

A meta-design model helps to support end-users to become co-designers. It enables them through different means to tailor their software environment in order to better adapt to changes in the underlying system. It supports the creation of dynamic environments that fits various contexts in which owners of diverse problems can act as designers. Instead of defining a finite product or content it rather provides the necessary tools that will enable end-users --according to their skills and culture-- to express a desired output. A meta-design model usually defines different modes of interaction allowing its users to perform different activities adapted to their own knowledge. (eg.  ``development mode'', ``design mode'', ``user mode''). Many meta-design models follow the Software Shaping Workshop approach.

Besides the meta-design model in itself, the question of how it can be transferred from one context of use into another --how the methods that have proven to work in one context can be reused in a systematic way in other contexts-- is opened research topic.~\cite{Carmelo2014}.