\subsection{VisTool}

VisTool is a Meta-Design model developed by Søren Lauesen that proposes an original approach to enable users to tailor their software and to become active members of the design team. It specifically emphasis on a systematic way to tailor and adapt user interfaces and visualize relational data.



Like other meta-design models, VisTool can be operated under a ``designer mode'' and a ``user mode''. The design mode allows end-users to adapt 

The current system is running under windows...

\subsubsection{A runtime bound to the local operating system}

Two types of source files are needed for generating a User Interface with VisTool. Vism files and Vis files. During the application's runtime, the vism file is opened first. The vismfile contains a reference to the initial vis file that is then opened.

This "chaining" of the source files causes the current system to be bound to the computer's operating system. A user provides the initial ``vism'' file available on the file system and uvis can trust that the link provided to the initial ``vis'' file is located on the same file system.