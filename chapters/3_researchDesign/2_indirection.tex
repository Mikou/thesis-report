With the emergence of web-based applications users have become familiar with new usability habits. They expect to navigate between HTML documents using the address bar of the browser and they expect these documents to automatically show all of the resources they need to access. It would go against common sense to prompt an user for a file as a necessary step to get access to an application. A main objective through porting VisTool into a browser is to make the system available for users independently of the device they are using. Prompting for a file would break the user's normal navigation experience. Actually, even if we wanted to implement the system in such a way, it would be very restrictive. For security reasons most browsers have strict access policies to the filesystem. HTML5 provides a filesystem API that allows the browser to access files on the local system in a ``sandboxed virtual file system''\cite{fsAPI}. At best we could implement an engine that requires the user to store all the system-dependent resources in that particular virtual file system. It would make it even less intuitive to use and this would be in full contradiction with the usability concerns that VisTool aims to improve.
\todo[inline]{Maybe here I could find a reference that discusses the emergence of usability habits in web based apps?}

Users will find it natural to use VisTool in a browser, only if it meets their usual habits. It should allow them to navigate to a vism file through the address bar of the browser and to get back an HTML document that presents an entirely usable interface with a content that corresponds to the activities that they should be able to perform. They will be able to easily use the application across devices and plateforms only if the resources that it depends on are not stored locally on a specific device on a specific plateform. User defined resources should be stored on a server.

The question arises then as to where the vism files and the vis files should be served from? The most direct and intuitive approach is to thinks of all the system and it's respective resources as being fetched from the same file server. This includes the HTML document --that the interface should be presented onto--, the visEngine, and the files used by the visEngine (vism and vis files). Such a solution would considerably limit the capabilities of the system to be adapted into different contexts of use. Indeed, it would imply that user specific files for all the different environments the system can be used in gets stored in one same location over which each environment has no control. This will certainly cause privacy issues. VisTool is an interface builder system that amongst other addresses healthcare environments. It is hard to imagine that a medical institution would accept to store data that can potentially give access to a third part to its entire patient records. It is in all cases unethical.

The system's resources should come from two distinct places. The HTML document along with the kernel should be accessible on a domain that belongs and is maintained by VisTool (I will refer to this domain as the VisTool domain) and the user related files should come from a destination that belongs and is maintained independently by the domain under which the application is used (I will use the term operator domain to refer to this).

The analysis of the relation between the vism file and the vis file, leads to the conclusion that VisTool can be developed as a user-friendly browser environment only if some sort of communication is established between the visEngine and the domain that uses it. The user will indirectly open a vis file through the vism file that he accesses. The operator domain owns the source files. They know how and where they can be accessed. Consequently, only they can provide the appropriate method for the engine to use in order to download these files. As I will show soon, the implementation of a mechanism that supports this indirection is surprisingly easy. However, the process of coming up with such an architecture has been much more complex than I initially thought it would.