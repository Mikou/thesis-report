\import{chapters/4_research/}{_intro}

\subsection{An architecture adapted to new usability conventions in the context of browsers}
\label{sec:indirection}
\import{chapters/4_research/}{1_architecture}

\subsection{Sample Application, a concrete Study Case}
\label{sec:studyCase}
\import{chapters/4_research/}{2_studyCase}

\subsection{VisEngine, Internal Communication, kernel API}
\label{sec:API}
\import{chapters/4_research/}{3_API}

\subsection{Components, Validators, Canvas}
\label{sec:canvas}
\import{chapters/4_research/}{4_canvas}

\subsection{Compilation process}
\label{sec:compilation}
\import{chapters/4_research/}{5_compilation}

\subsubsection{Tokenizer}
\label{sec:tokenizer}
\import{chapters/4_research/}{5.1_tokenizer}

\subsubsection{Parsers}
\label{sec:parsers}

The system uses two distinct parsers: One for parsing vism files another for parsing vis files. As I said, both parser's are using the same tokenizer. Each parser is a Javascript module that exposes a similar interface containing:
\begin{itemize}
    \item An ``init'' method that allows to initialize the module. The tokenizer and other modules that the parser depends on can thereby be injected.
    \item A ``Parse'' method that triggers the actual parsing process.
\end{itemize}
Listing~\ref{lst:initVisParser} shows a concrete example of how the visParser is initialized and used. 

\lstinputlisting[
    caption={Example of how the visParser is configured and used},
    label={lst:initVisParser}
]{code/initVisParser.js}

\subsubsection{VismParser}
\label{sec:vismParser}
\import{chapters/4_research/}{5.2_vismParser}

\subsubsection{VisParser}
\label{sec:visParser}
\import{chapters/4_research/}{5.3_visParser}

\subsection{Interpretation}

\subsection{Compilation flow}

Before getting into the detail of the interpretation, a brief recall of the entire program flow will hopefully help the reader to easier follow.

The consumer application starts by configuring the system in order to pass the resource provider method and the placeholder (DOM element) that the interface will be injected into.

The application is then started by the consumer application by invoking the \texttt{uvis.openVismFile('<filename>')} method from the public API (in the file~\texttt{kernel/src/main.js}). The entire kernel is reset. This is useful when an interface was already generated, in order to unset all the objects (template list etc.). The content of the vismfile is asynchronously retrieved. Then, the vismParser parses the vismfile. Depending on whether or not the vismfile describes a database or not, a database schema will be built, the content of the visfile whose name was retrieved from the vismfile will be asynchronously downloaded.

The kernel will then try to get the data instances required for evaluation. The pre-evaluation phase reorganizes the templates into a template tree and prepares the set of queries that will be executed in order to retrieve these instances.

Then, when the data instances have been downloaded, the interface can be rendered. This is where the evaluation phase comes in. It computes the expression trees in each formula into an actual value in order to assign it to the component type addressed for each template.

The pre-evaluation and evaluation phases share a code structure that is quite similar. What really changes is the semantic meaning given to the nodes of type path in both of these contexts. In the pre-evaluation phase they will help to construct or find the allocation spaces and fill them with appropriate values, in the evaluation phase they will serve the purpose of retrieving and consuming the data.

\subsection{Pre-evaluate}
\label{sec:PreEvaluate}

\lstinputlisting[
    caption={function describing how to get the data instances needed for the evaluation},
    label={lst:getInstances}
]{code/allocation/getInstances.js}

Listing~\ref{lst:getInstances} shows how the system recovers the data instances. It shows also that it uses the \texttt{preEvaluation} function. That function takes two input parameters. An expression and an environment. The expression is the node tree to be pre-evaluated, the formula. The environment, is an object that represent the scope, the set of addressable things during the pre-evaluation.

What listing~\ref{lst:getInstances} also tells us is that \texttt{{preEvaluate}} is first called for evaluating the formula relative to the \texttt{Rows} property and then, it is called as many times as the total number of properties in the template. Another implementation style could possibly split the preEvaluation function into two distinct functions. One exclusively used for pre-evaluating the \texttt{Rows} formula, another one to pre-evaluate the rest of the properties. The consequence of having these functions as a single one is that the \texttt{preEvaluation} function needs to perform additional tests to determine what type of property it is dealing with, when entering a node in the syntax tree of the formulas. On the other hand, it makes the code more compact.

The reason why rows is first called for all templates is because it will help to build the template tree and to attach the addressed resources from the data map onto the appropriate template they belong to, into their ``query model''.

The query model is the object on a template that represents the query that will be translate into a query string.

When calling the preEvaluate once again for all other properties for all templates, the preEvaluate function will simply be used in order to add the addressed resource properties into the appropriate query model.

\lstinputlisting[
    caption={pre-evaluation function},
    label={lst:preEvaluate}
]{code/allocation/preEvaluate.js}

Within the pre-evaluate function, a relatively classic code structure encountered in interpreters, allows the program to navigate amongst the different nodes of the syntax tree of a formula. This is what listing~\ref{lst:preEvaluate} shows. 

When a node of type formula is read from a template property, a recursive call to \texttt{preEvaluate} is made with the value of the expression contained in the formula. 

When a node of type binary is read, a recursive call is made to \texttt{preEvaluate} for both the left expression and the right expression. Finally, if currently pre-evaluating the \texttt{Rows} property, the value of the operator is applied to both of these expressions. It is unnecessary to apply the operator for any other property type since we do not need to actually evaluate the expressions of the template properties yet.

When an atomic expression is read (num, str, punc, datetime), the value of that expression is simply returned. In JavaScript all objects inherits a \texttt{toString} function. Here I invoke that function to make sure that dates (which are built using moment.js) are transformed into readable strings, in this way, it will be easy in the future to generate dates according to a given culture that could be configured within the system.

What is really specific to this pre-evaluation function is how I enable the walk principle I have mentioned earlier. It relies on a cooperation between what is read in a node of type ``path'' and what is returned from reading a node with the atomic type ``identifier''. This deserves a subsection on it's own.

\subsubsection{Walk principle}

Recall that the parser generates a ``path'' node that contains a nested set of components. The value of a component is accessible through the property \texttt{content} and the next component (if any) is accessible from the \texttt{next} attribute. Listing~\ref{lst:path} showed a concrete example of such a node.

\lstinputlisting[
    caption={create path reader function},
    label={lst:createPathReader}
]{code/allocation/createPathReader.js}

When reading a node of type \texttt{path}, the expression is first encapsulated into a \texttt{pathReader} which is a helper function returning an object through which the given path can easily be navigated through. Listing~\ref{lst:createPathReader} show how this helper is created.

For an expression like \texttt{Map.Person[3]!Name} a first call to the \texttt{next} method of the path reader will return the content of the first object in the path chain, in this case, the value \texttt{\{type:'id’, value:’Map’\}}. A new call to that same method will return the content of next component in the path chain: \texttt{\{type:'punc’, value:’.’\}}. The next method can be called indefinitely. When a path has be read all the way through, the \texttt{next} method of the reader will simply return null. The \texttt{hasNext} method can helps to determine if there is a next element in the path.

When the path reader is ready, a \texttt{walkTo} function is evaluated, based on the value of the expression contained in the first component of the path (in the case of the path expression \texttt{Map.Person[3]!Name}, this would be the value \texttt{Map}).

When the \texttt{walkTo} function is resolved, it is called and the path reader along with the environment, are passed as it's input parameters. In this way, no matter what the function walkTo is, it can internally process anything it needs by using the path reader to advance in the current path and the environment to find the objects it needs to address.

The function assigned to the \texttt{walkTo} variable is determined when the JavaScript engine reads an atomic expression of type \texttt{id}. It will check if the value is \texttt{index}. In the pre-evaluation context, \texttt{index} has no semantic meaning, null is just returned, It will then check if the value is equal to one of the reserved names associated to the existing set of walkTo functions. If it is none of those, the value of the expression is just returned as is.

\subsubsection{Inferring the type of walkTo function}

The system's requirements specifies that designers should be allowed to omit the path prefix (that first component in a path that helps to determine the walkTo function). The expression \texttt{Map.Person[3]!Name} would then become \texttt{Person[3]!Name}.

It would be fairly easy to add support for such an omission into the web-based implementation. It is a matter of adding an extra function before returning an identifier's value when reading a node of type identifier. That function could then use the environment to lookup the ``unknown'' identifier's value in all of the possible address spaces. If the value is found in one of them, the walkTo function to return should be the one that allow to move into that address space. If the value is found in more that a single addressing space, then, the system should warn the designer that his choice is ambiguous.

\subsubsection{Building the template Tree}

The template tree is built when applying the pre-evaluate function on the Rows of all templates. In this case, when reading a binary node, the value of the operators will be applied onto the operands of the expression since the condition \texttt{env.property === 'Rows'} will be satisfied.

An expression like \texttt{Form.txtPerson -< Map.activity WHERE Form.txtPerson.Name = 'Bob'} will cause the resource \texttt{activity} (found in the data map) to be stored in the query model of the template with name \texttt{txtPerson}.


\subsection{Evaluate}
\label{sec:Eval}
